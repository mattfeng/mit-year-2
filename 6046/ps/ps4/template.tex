\documentclass{6046}

\author{Matthew Feng}
\problem{4}
% \problem{A} means Problem Set A.
\collab{Eric Qian, Tyler Barr, Alex Guo}
% or give names, e.g., \collab{Alyssa P. Hacker and A. Student}

\begin{document}

\section*{Problem 1}

\subsection*{(a)}
Since we must first prioritize Jane's commute, we first
compute the shortest path from Jane's office $o$ to her house $h$.
We can do this $O(E + V\log V)$ time using Dijkstra's algorithm
with Fibonacci heaps. After we have found this path,
we create a supernode $z$, and connect $z$ to every
node in that path with an edge of weight $0$, and then
delete the edges in the original graph. We can
then run Prim's algorithm starting with node $z$ to find
the ``minimum spanning tree'' that also includes the
shortest route from $o$ to $h$ (in reality, we 
are finding the minimum spanning trees for each of
nodes in the shortest path and then connecting them
via the edges in the shortest path). Finally, we remove
the $0$ weight edges and add back the original shortest path
edges and return that set of edges. The algorithm will
find the minimum cost edges to form the minimum spanning
tree because after we remove the shortest path edges
from the graph and then run the MST algorithm, we will
find a true MST for the graph. By re-adding the shortest path edges,
we ensure that Jane has her shortest path, and the rest of
the MST has minimum cost paths given the requirement that
Jane's shortest path must be in the graph. Prim's algorithm
also runs in $O(E + V\log V)$ time with a Fibonacci heap,
so the overall algorithm will run in $O(E + V\log V)$ time.

\subsection*{(b)}
We can build off of our start from the previous part and only
consider the edges in our pseudo-MST as well as the $k$ edges
with reduced cost. We want to reuse as much of our previous MST
as possible, which consists of pieces that we are sure will be
part of our new MST $T''$. We first break up $T'$ along any of
$k$ edges that have reduced values. We know that each of the
edges in the disjoint sets that we have now separated $T'$ into
(including the disjoint set containing shortest path nodes,
because we include a supernode that connects all of those with
edges of weight $0$, so that they will definitely be part of
the MST) will be part of a MST, because of the {\bf optimal substructure
property}. After we sort the $k$ edges in $O(k\log k)$ time, we can use
Kruskal's algorithm to reconnect the sub-MSTs considering only the
$k$ reduced cost edges, because of the {\bf cut
property} --- the light edges that cross the various cuts consisting
of $(S, V - S)$ where $S$ is a disjoint set created from when we broke
up our original MST must be among the $k$ edges, else they would
not have been broken (or would have been part of the original MST).
Since we only look at $k = O(V)$ edges when running Kruskal's algorithm, 
Kruskal's algorithm runs in $O(V\log V)$ time, and thus the total
running time of our algorithm is $O(V\log V)$.

\section*{Problem 2}
\subsection*{(a)}
We construct the digraph $G = (V, E)$ with capacity function
$c(u, v): E \rightarrow \mathbb{R}$ denoting the
maximum flow that can pass through edge $(u, v)$ at
any moment in time.

This is a variation on the bipartite matching problem.
As such, we have $m$ vertices $r_j \in V, \forall
j \in [1, m]$ that will represent riders waiting to be
matched, and $n$ vertices $k_i \in V, \forall i \in
[1, i]$ that represent cars that
can hold riders. We have a source $s$ and
sink $t$ as well, which will allow us to think
of the flow in this network intuitively as riders
getting (matched) into cars.

We then will construct $m$ edges $(s, r_j)$, $\forall j \in [1, m]$,
where $c(s, r_j) = 1$, as each rider only needs one car.

We will also construct $n$ edges $(k_i, t)$, $\forall i \in [1, n]$,
where $c(k_i, t) = s_i$, the seat capacity of each car. By setting
the capacities in this way, we can sure that each car is matched with
at most its seat capacity.

Finally, for each rider $j$, we construct edge $(r_j, k)$, $\forall k \in C_j$,
each with infinite capacity. 

\subsection*{(b)}
We make $H + 1$ copies of the graph $G$, indexed as $G^i = (V^i, E^i)$,
where the superscript represents the number of hours
that have elapsed (thus $i \in [0, H]$). For every airport $v^i_n \in V^i$, we
construct edges $(v^i_n, v^{i + h_j}_m)$ with capacity $c_j$ for all flights $e_j$
starting from airport $v_n$ and landing in airport $v_m$.
We also construct edges $(v^i_n, v^{i + 1}_n)$ with infinite capacity to represent
passengers who wait at the airport for their flight. Then, we connect
all Boston airport destination nodes $t^i$ to a true sink node $t^*$, where each
of those connecting edges has infinite capacity as well. Finally, we connect
a source node $s$ to $v^0_n$ for every $v_n \in V$ with edge $(s, v^0_n)$ with
capacity $c_n$, representing the number of people starting off from that airport.
If we run max flow on this graph, we will be able to find the maximum of
passengers that Ryde can bring to Boston, because we have encoded every possible
travel path in our flow network.

\subsection*{(c)}
Since our graph will have $HV + 2$ nodes ($H$ copies of $V$ nodes, as well as $s$ and $t*$),
and $HE + V + H$ edges ($H$ copies of $E$ edges, with $V$ edges
connecting $s$ to all vertices in $V^0$, and $H$ edges connecting $t^i$ to $t^*$),
for a total runtime of $O((HV + 2)(HE + V + H)^2)$, or $O((HV)(HE + V + H)^2)$.

\subsection*{(d)}
We can run DFS on the residual graph $G_f$ twice.
We first run DFS once from the source $s$, keeping the directions
in $G_f$ intact, so that we can find all the nodes that are part of
the connected component that $s$ belongs to. We then flip the directions of
the edges in the residual graph, and run DFS from $t$, so that we can find
all the nodes in the connected component containing $t$. We find the
connected components for $s$ and $t$ so that we can find the augmenting paths
from $s$ to $t$ that are missing a single edge. We know that $s$ and $t$ must be in
separate connected components, else the flow that we had found would not have
been a max flow (because there would have been an augmenting path). For every node $u$ that is in $s$'s
connected component, we want to see if we can add a flight to some node $v$ in $t$'s
connected component --- if so, then that is a valid flight to add. DFS runs in $O(H(V + E))$
time. Since each connected component can have at most $O(HV)$ nodes, and we want to
compare all possible pairs, finding all possible edges that might have a valid flight
takes $O(H^2V^2)$ time, so the total runtime of the algorithm is $O((HV)^2)$.

\end{document}

