\documentclass{6046}

\usepackage{amsmath}
\usepackage{enumitem}

\author{Matthew Feng}
\problem{6}
\collab{James Lin, Katherine Xiao}

\begin{document}

\section*{Problem 1}
\subsection*{(a)}

If there are no cycles in $G$, that means that the number of
edges in every connected component of $G$ is one less than
the number of nodes in that connected component. That means
that there more valid slots available than items we want to insert,
and thus it is possible to place the items into the one of
the two slots that $h_1$ and $h_2$ assign them to.

In particular, for every connected component $\kappa$ in $G$,
perform DFS from an arbitrary start vertex in $\kappa$.
Every edge $e_i$ corresponds to some value $x_i$ that we wish
to put into our hash table. Every time we traverse down
an edge $e_i = (u, v)$, we assign the $x_i$ corresponding to the
edge to the vertex that we arrive at (i.e. $v$, if we started
from $u$). Note that this means the start vertex (slot) we chose
will not be assigned a value.

The algorithm is correct because DFS traverses every edge
exactly once, meaning that every value we want to insert is
assigned to a slot that it hashed to. Moreover, each vertex is
visited exactly once, and so each vertex (slot) will never
be assigned to two values (i.e. no values will ever be overridden).
Since every value is given a slot, and no slot is ever used twice,
the algorithm is correct. Furthermore, DFS runs in $O(n)$ time,
so the runtime of our algorithm is $O(n)$.

\subsection*{(b)}

For a given $k$-cycle defined by edges
$C = \{(m_1, m_2), (m_2, m_3), ..., (m_{k-1}, m_k)
, (m_k, m_1)\}$ (noting all edges are undirected,
and the cycle occurs in that particular order),
the probability that an edge $e = (m_a, m_b) \in C$ exists
is $2/m^2$, since the edge will exist if either
\begin{enumerate}[noitemsep,nolistsep]
    \item $h_1(x) = m_a$ and $h_2(x) = m_b$, or
    \item $h_1(x) = m_b$ and $h_2(x) = m_a$,
\end{enumerate}
both of which have a probability of $1/m^2$ individually
of occuring. Since $x$ can be any of the $n$ items, the 
probability that $e$ exists is $2n/m^2$. We need
all $k$ of the edges in $C$ to exist, and each may appear
independently, and so the probability that all exist
is $(2n/m^2)^k$.

\subsection*{(c)}

Since there are at most $\frac{(k - 1)!}{2}\,\binom{m}{k}$ potential ways
for a $k$-cycle to exist in $G$ ($\binom{m}{k}$ ways to choose
the vertices, and $\frac{(k - 1)!}{2}$ ways to order the cycle), the probability that
of any $k$-cycle appearing in $G$ is upper bounded (union bound) by 
$$\frac{(k - 1)!}{2}\,\binom{m}{k}\left(\frac{2n}{m^2}\right)^k \le
\left(\frac{2n}{m}\right)^k.$$

\subsection*{(d)}

From (c), we know that the probability of any $k$-cycle is
at most $(\frac{2n}{m})^k$. Then the probability of
any cycle is upper bounded (again by union bound) by
$$\sum_{k = 1}^{\infty} \left(\frac{2n}{m}\right)^k.$$

If $m = 10n$, then the above sum becomes
$$\sum_{k = 1}^{\infty} \left(\frac{1}{5}\right)^k
= \frac{1}{4} < \frac{1}{2}.$$

\subsection*{(e)}
If we run our algorithm in (a) multiple times (each time
generating new hash functions $h_1$ and $h_2$), where each
will succeed with probability of at least $\frac{1}{2}$, then
the chance that Ben fails decreases exponentially. Since
each run has at least $\frac{1}{2}$ chance of succeeding,
the expected number of runs before a success (including
the success) is $2$ (by expectation of a geometric
random variable). Since the expected number of needed runs is
constant, and it takes $O(n)$ time for each call to the
algorithm, the expected running time of this hashing
procedure is $O(n)$.

\section*{Problem 2}
\subsection*{(a)}
By definition, a zero-sum game requires
$F(x, y) + G(x, y) = 0$.

\subsection*{(b)}
Beff always has a strategy to ensure that Melon's payoff is
$\le \frac{1}{3}$.

We can split all of Melon's potential strategies into two cases:
\vspace{-1em}
\paragraph{Case 1.} {\it The probability that $x < \frac{1}{2}$ is
less than or equal to $\frac{1}{3}$}.\\
\vspace{-1em}

In this case, if Beff employs the strategy $y = 1$, then
Melon's expected payoff is 
\[
    (\text{some value} \le \frac{1}{3})
  - (\text{some value} \ge \frac{2}{3})
\]
which is less than $\frac{1}{3}$.

\paragraph{Case 2.} {\it The probability that $x < \frac{1}{2}$ is
greater than $\frac{1}{3}$}.\\
\vspace{-1em}

In this case, if we let $y = a$ such that
$\mathbb{P}[x < a] = \frac{1}{3}$, then the maximum expected payoff that Melon
can achieve is 
\[
    (\frac{2}{3})
  - (\frac{1}{3})
\]
where the former term is the positive payoff and the latter
is the negative payoff. This
payoff sums to $\frac{1}{3}$. Note that $y < \frac{1}{2}$ because
in order for $\mathbb{P}[x < \frac{1}{2}]$ to be greater than
$\frac{1}{3}$, it must have equaled $\frac{1}{3}$ for some
$x < \frac{1}{2}$.

In both cases, which span all of Melon's strategies, 
Beff has a response that guarantees Melon's payoff is at most $\frac{1}{3}$.

\subsection*{(c)}
This time, we want to find ways to choose $x$ so that for any
probability distribution on $y$, Melon has a payoff of at least
$\frac{3}{7}$. We again proceed with cases on the
strategies that Beff could use.
\vspace{-1em}

\paragraph{Case 1.} {\it
The probability that $y < 1$ is less than $\frac{3}{7}$.
}

This case needs to be further divided into the following two cases:
\vspace{-1em}

\paragraph{Case 1a.} {\it
The probability that $y < \frac{1}{2}$ is
less than or equal to $\frac{1}{7}$}.\\
\vspace{-1em}

In this case, if Melon always selects $x = 0$, then his expected
payoff will be 
\[
    (\text{some value} >\frac{4}{7})
  - (\text{some value} \le \frac{1}{7})
  + (\text{some value} \ge \frac{2}{7})
\]
which is guaranteed to be greater than or equal to $\frac{3}{7}$
(the latter two terms always sum to a non-negative value).

\paragraph{Case 1b.} {\it
The probability that $y < \frac{1}{2}$ is
greater than or to $\frac{1}{7}$}.\\
\vspace{-1em}

Pick $x = b$ for the $b$ such that
$\mathbb{P}[y < b] = \frac{1}{7}$. $x$ is guaranteed
to be less than $\frac{1}{2}$ because in order for
$\mathbb{P}[y < \frac{1}{2}]$ to exceed $\frac{1}{7}$,
$\mathbb{P}[y < \frac{1}{2}]$ must have been equal to $\frac{1}{7}$
(since the cumulative distribution is monotonically increasing).
Then, Melon's expected payoff is 
\[
    (\text{some value} >\frac{4}{7})
  - (\text{some value} < \frac{2}{7})
  + (\frac{1}{7})
\]
which is also guaranteed to be greater than or equal to $\frac{3}{7}$.
The first term comes from when $y = 1$, the third from the assumption
in the case, and the second is the worst case in that the remaining
distribution of $y$ is in the $-1$ ``zone''.

\paragraph{Case 2.} {\it
The probability that $y < 1$ is greater than or equal to $\frac{3}{7}$.
}
\vspace{-1em}

In this case, if Melon always selects $x = 1$, then his payoff will be
\[
    (\text{some value} \ge \frac{3}{7})
  - 0 \times (\text{some value} \le \frac{4}{7})
\]
which is guaranteed to be greater than $\frac{3}{7}$.

Because the cases cover all possible strategies that Beff could pick,
Melon always can respond with a strategy that guarantees him a 
payoff of at least $\frac{3}{7}$.

\subsection*{(d)}
No, since the minimum payoff for Melon if Beff chooses his strategy first
does not equal the maximum payoff he can ensure if Beff chooses his
strategy second.

\subsection*{(e)}
This is not a Nash Equilibrium, because either Melon or Beff can change
their strategy (e.g. $x = 0$) to gain a higher payoff.

\end{document}

