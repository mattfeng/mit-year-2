\documentclass{6046}

\author{Matthew Feng, R06}
\problem{1}
% \problem{A} means Problem Set A.
\collab{Lauren Oh, Lillian Bu}
% or give names, e.g., \collab{Alyssa P. Hacker and A. Student}

\begin{document}

\section*{Problem 1-1}
\subsection*{(a)}
$$T(n) = 5T(\frac{n}{4}) + n$$
\paragraph{Answer}
$T(n) = \Theta(n^{\log_4 5})$
\paragraph{Solution}
This falls under case 1 of the Master Theorem, which states that
if $f(n) = O(n^{\log_b a-\epsilon})$ for some $\epsilon > 0$,
then $T(n) = \Theta(n^{\log_b a})$. In this case, $a = 5$, $b = 4$,
and $f(n) = n$. Since $n = O(n^{\log_b a - \epsilon})$
for $\epsilon = n^{\log_b a} - 1$, we have $T(n) = \Theta(n^{\log_4 5})$.

\subsection*{(b)}
$$T(n) = 9T(\frac{n}{3}) + n^2$$
\paragraph{Answer}
$T(n) = \Theta(n^2\log{n})$
\paragraph{Solution}
This falls under case 2 of the Master Theorem, which states that
if $f(n) = \Theta(n^{\log_b a})$, then $T(n) = \Theta(n^{\log_ba}\log{n})$.
In particular, $a = 9$, $b = 3$, and $f(n) = n^2$.
Since $n^2 = \Theta(n^{\log_3 9}) = \Theta(n^2)$, $T(n) = \Theta(n^2\log{n})$.

\subsection*{(c)}
$$T(n) = T(n - 2) + n$$

\paragraph{Answer}
$T(n) = \Theta(n^2)$.
\paragraph{Solution}
For this problem, we do not use the Master Theorem, but instead rely on
the tree method. In particular, we create a tree that has $O(\frac{n}{2})$
layers, since at every layer we decrease $n$ by $2$, and the cost per layer
is the size of that layer. The total cost then is
$$T(n) = n + (n - 2) + (n - 4) + \text{\textellipsis} + 4 + 2 =
\frac{n}{2}(\frac{n}{2} + 1)$$
and so we find that $T(n) = \Theta(n^2)$.

\subsection*{(d)}
$$T(n) = 3T(\sqrt{n}) + \log{n}$$

\paragraph{Answer}
$T(n) = \Theta((\log n)^{\log_2 3})$
\paragraph{Solution}
To solve this problem, we use the Master Theorem after involving a change
of variable, since we know that we can turn powers into multiplication
via logarithms. Concretely, let $m = \log(n)$. Then we
have $T(2^m) = 3T(2^{m/2}) + m$. Let $S(m) = T(2^m)$.
Then $S(m) = 3S(\frac{m}{2}) + m$, which by case 1 of the Master Theorem, 
dictates that $S(m) = T(2^m) = \Theta(m^{\log_2 3})$. Switching our
variable back to $n$, we get $S(\log n) = T(n) = \Theta((\log n)^{\log_2 3})$.

\subsection*{(e)}
$$T(n) = 2T(\frac{n}{4}) + T(\frac{n}{3}) + \Theta(n)$$

\paragraph{Answer}
$T(n) = \Theta(n)$
\paragraph{Solution}
We will use the Akra-Bazzi method to solve the problem. First, we notice that
$0 < p < 1$, since $Q(p) = 2(\frac{1}{4})^p + (\frac{1}{3})^p$ is a decreasing function,
and $Q(0) = 3 > 1 > \frac{5}{6} = Q(1)$. Akra-Bazzi tells us that
$$T(n) = \Theta\left(n^p\left(1 + \int_1^n\frac{f(u)}{u^{p+1}}du\right)\right)$$
which, when applied to this recurrence, yields
$$T(n) = \Theta\left(n^p\left(1 + \int_1^n \frac{cu}{u^{p+1}}du\right)\right)$$
for some positive real number constant $c$.
Completing the calculation of the integral,
\begin{align}
T(n) &= \Theta\left(n^p\left(1 + \int_1^n \frac{cu}{u^{p+1}}du\right)\right)\\
     &= \Theta\left(n^p\left(1 + c \int_1^n u^{-p}du\right)\right)\\
     &= \Theta\left(n^p\left(1 + \frac{cn^{1-p}}{1 - p}\right)\right)\\
     &= \Theta(n)
\end{align}

\subsection*{(f-1)}
$\log \log n = o(\log n)$, because the limit as $n
\longrightarrow \infty$ of $\dfrac{\log n}{\log \log n}$
is unbounded.\\
$\log{n^2} = \Theta(\log_5 n) = \Theta(\log n)$, because of rules of logarithms (change of base, exponents inside logarithms).\\
$\log n = o(\log^2 n)$, because the limit as $n
\longrightarrow \infty$ of $\dfrac{\log^2 n}{\log n}$
is unbounded.\\
$\log^2 n = o(\sqrt{n})$, because the limit as $n
\longrightarrow \infty$ of $\dfrac{\sqrt{n}}{\log^2 n}$
is unbounded, as determined by L'Hopital's rule.\\

\subsection*{(f-2)}
$n \log n = \Theta(\log n!)$, by both Stirling's Approximation and
the rules of logarithms ($\log ab = \log a + \log b$).\\
$n \log n = o(n^{4/3})$, because the limit as $n
\longrightarrow \infty$ of $\dfrac{n^{4/3}}{n \log n}$
is unbounded, as determined by L'Hopital's rule.\\
$n ^ {4/3} = o(n^2)$, because the limit as $n
\longrightarrow \infty$ of $\dfrac{n^ {4/3}}{n^2}$
is $0$.\\
$n ^ 2 = o(2^n)$, because the limit as $n
\longrightarrow \infty$ of $\dfrac{n^2}{2^n}$
is $0$.\\

\subsection*{(f-3)}

$2^{{\log n}^{\log \log n}} = o(e^n)$,
because the limit as $n \longrightarrow \infty$
of $\dfrac{2^{{\log n}^{\log \log n}}}{e^n}$ is $0$, which
we can see by first taking the logarithm of the limit
and noticing that $n > {\log n}^{\log \log n}$ for large $n$.\\
$e^n = o(n^{n-1})$, because the limit as $n \longrightarrow \infty$
of $\dfrac{ne^n}{n^n}$ is $0$\\
$n^{n-1} = o(n^n)$, because the limit as $n \longrightarrow \infty$
of $\dfrac{1}{n}$ is $0$.\\

\section*{Problem 1-2}
\subsection*{(a)}
Let every person in the group $G$ representing the MIT Class of 2050 be
denoting $p_i$, for $i \in [1, n]$. Let $G_1$ represent the group of
students that belong to the same universe as $p_1$, and $G_x$ represent
those who do not belong the same universe as $p_1$. If we ask queries
``Are person $p_1$ and $p_i$ from the same universe?'' for all $i
\in [2, n]$, then we can determine the members of $G_1$ and $G_x$, which
are mutually exclusive sets by construction. The larger group must be those
from Earth-X because there are more X-ers than Earthers, and everyone in that
group must be from the same universe, due to transitivity of equivalence.

Since we make $O(n)$ queries each taking $O(1)$ time, the total running time
this algorithm is $O(n)$.

\subsection*{(b)}
We will show that it is impossible to determine the group of Earthers
if X-ers can lie and there are more X-ers than Earthers. We will do this by
construction.

First, we notice that if we ask the same two people $(A, B)$ for their
response, they will always respond in the same way because switching answers
gives away whether or not one is an X-er.

After this observation, we realize that the maximum number of meaningful
queries we can make is $n(n - 1)/2$, one query for every possible pairing.
We aim to demonstrate that the X-ers can always respond in a way that makes
it impossible to determine, because of symmetry, whether or not a subgroup
is from Earth or from Earth-X.

Essentially, if the X-ers work to maintain that the number of $(E, E)$ responses
they give is the same as the number of $(E, E)$ responses that real
Earthers give, then Ben is unable to distinguish between a fake $(E, E)$ response
and a real $(E, E)$ response, meaning that Ben won't be able to confidently
create a group of all Earthers. In particular, the X-ers can always accomplish
this, because the number of real $(E, E)$ responses that can possibly be given
is fewer than the number of fake $(E, E)$ responses that can be given (because
there are more X-ers than Earthers).

As such, the X-ers can always respond in a way that makes then indistinguishable
from Earthers because of symmetry, and thus Ben cannot determine who is from Earth.

\subsection*{(c)}
Let $(a, b)$ be the response to query $(A, B)$, where $a, b \in {E, X}$,
and $a, b$ correspoonding to the response that $A, B$ gave about the 
other person, respectively. Then, there are three unique responses: 
\begin{enumerate}
\item $(E, E)$, where $A$ says $B$ is from Earth, and $B$ says $A$ is from Earth.
\item $(E, X)$, where $A$ says $B$ is from Earth, and $B$ says $A$ is from Earth-X.
\item $(X, X)$, where $A$ says $B$ is from Earth-X, and $B$ says $A$ is from Earth-X.
\end{enumerate}

For response 1, there are either 0 X-ers or 2 X-ers. If both $A$ and $B$ 
are from Earth, then they would respond truthfully with $(E, E)$. If $A$ and $B$
are both from Earth-X, then they can collaborate and lie to give the
response $(E, E)$. However, if one is from Earth and the other is from
Earth-X, then the person from Earth would be truthful and respond with $X$,
which is impossible because we are only considering the case of the response
$(E, E)$.

For response 2, there are either 1 X-ers or 2 X-ers. There cannot be 
0 X-ers, because then the response would be $(E, E)$, not $(E, X)$. One X-er
is possible because the one from Earth will respond with $X$ truthfully, 
while the X-er can be truthful as well as respond with $E$. Two X-ers is
also possible because one may tell the truth and the other can lie and give
the response $(E, X)$.

For response 3, again there are either 1 or 2 X-ers. It is impossible to have 
0 X-ers, because then the response would be $(E, E)$, not $(X, X)$. 1 X-er 
is possible because the Earther will respond truthfully with $X$, while the X-er
may lie and respond with $X$ as well. Likewise, 2 X-ers is possible, because
they can both decide to be truthful and both say that the other is from Earth-X.

\subsection*{(d)}

One of the key insights to solving the problem is to realize that
once we have identified a single person is from Earth, we are done,
because we may use that person as an ``oracle'' who will tell us the true
identity of every other person in the group.

In order to accomplish this, we will maintain the invariant that in our group
$G$ of students we are focusing on, {\bf there will always be more
Earthers than X-ers}. Why? Because if we can whittle down the size of the
group we are considering while still maintaining this invariant, then if
we reduce the size of the group to a single person, we are done: the 
last one standing must be from Earth.

We will thus proceed as follows: pair every student with another student
as $(A_i, B_i)$, $i \in [1, \frac{n}{2}]$, leaving one student by 
himself/herself if $n$ is odd. Then, ask each pair $(A_i, B_i)$ for their 
response to the question as described in part (b).

The second key insight is as follows: if the response to $(A_i, B_i)$ is
either $(E, X)$ or $(X, X)$, you may immediately remove them from the
group $G$ of students we are focusing on while still maintaining
the invariant. This is because there is at least 1 X-er in the pair, and that
we begin with more Earther than X-ers, so removing one of each still maintains
that there are more Earthers and X-ers in group $G$. Of course, if both
students in the pair turn out to be X-ers, then we still will have maintained
the invariant, because only the number of X-ers will have decreased.

The more difficult case is when the response to $(A_i, B_i)$ is $(E, E)$.
Then, either there are 0 X-ers, or there are 2. The final insight is to realize
that this forces the identity of $A_i$ and $B_i$ to be the same: either they
are both X-ers, or they are both Earthers. Thus, we may immediately remove either 
$A_i$ or $B_i$ from group $G$ (or, in other view, merging $A_i$ and $B_i$
into a single person) while still maintaining the invariant, because if 
there were more Earthers beforehand, then there must be more Earther/Earther pairs 
than X-er/X-er pairs, so removing one person from every pair will still leave
more Earthers than X-ers. In this way, we are able to consider every possible
response, and strictly decrease the size of the group at every iteration,
which means that we will always terminate with a single Earther remaining.

Finally, once we have identified a single Earther, we interrogate every other
member of the class of 2050 using this Earther and determine whether
or not the other member is from Earth.

The running time of this algorithm is given by the recurrence
$T(n) = T(n / 2) + O(n)$, because at each iteration, we need to make $O(n)$
queries each of $O(1)$ time, and in the worst case reduce the size
of the problem by half (if every pair answers $(E, E)$). The solution to
this recurrence is $\Theta(n)$ by case 3 of the Master Theorem. Adding the
additional $\Theta(n)$ time needed to interrogate every other student using
the Earther, the final runtime complexity of our algorithm is $\Theta(n)$.

\end{document}

