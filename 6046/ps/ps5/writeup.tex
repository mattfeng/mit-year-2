\documentclass{6046}

\usepackage{mathtools}

\author{Matthew Feng}
\problem{5}
\collab{Lauren Oh, Lillian Bu}

\begin{document}

\section*{Problem 1}

\subsection*{(a)}
We wish to maximize the revenue $\sum_{i = 1}^{n}c_ix_i$
that Melon makes subject to the constraint(s) that he cannot
use more materials than he has. Concretely, those two
statements translate into the following equations
(in standard form):
\begin{displaymath}
\begin{matrix*}[l]
\max              & {\displaystyle \qquad \sum_{i = 1}^{n}c_ix_i }\\
\vspace{-0.5em}\\
\text{subject to} & {\displaystyle \qquad \sum_{i = 1}^{n}a_{ij}x_i \le b_j},
                  & \forall j           &\hspace{-0.5em}\in [1, n]\\
\vspace{-0.5em}\\
                  & \qquad x_i \ge 0,
                  & \forall i           &\hspace{-0.5em} \in [1, n]
\end{matrix*}
\end{displaymath}

where $x_i$ represents the (not necessarily integral) number
of type $i$ vehicles to manufacture.

\subsection*{(b)}
The general idea behind getting a $O(\log P^*)$ time algorithm
is to binary search on the value of $P$. Specifically, we want
to introduce an additional constraint for our friend such that
the revenue $P$ is greater than some amount $P'$. We will have
found $P*$ when we find the value of $P'$ such that for any value
$P'' > P'$, adding the constraint $P > P''$ results in an
infeasible LP.

Specifically, let us define a value $P$ that represents the
minimum revenue that we want Melon to make. Then we can add
this constraint to the constraints we have listed in part (a)
as
\[
    \sum_{i = 1}^{n}c_ix_i \ge P
\]
where $P$ is a constant value that we control. Let us start
with $P^{(0)} = 1$, where the superscript represents the
iteration index that we are on. If we hand this LP to our
MIT oracle, then we will see whether or not the LP is feasible.
If the LP is feasible, double the value of $P$ so that
$P^{(k)} = 2^k$ (i.e. $P^{(k)} = 2P^{(k - 1)}$). We continue
until the problem becomes infeasible, meaning that there
is no combination of numbers of vehicles to manufacture that
can result in such a return $P^{(K)}$. Then $P^*$ must be
between $P^{(K - 1)}$ and $P^{(K)}$. We can binary search
on this remaining interval (which is guaranteed to be
of size $O(P^*)$) until we find the exact $P'$ such that
$P'$ is feasible, but any value greater than $P''$ makes the
LP infeasible. That $P'$ is $P^*$, the maximum amount that Melon
can make.

Binary search is viable because the predicate ``Melon can make
more than $P$ in revenue'' is true for all values $\le P^*$,
and false for any value $> P^*$. Furthermore, because we bounded
the value of $P*$ with $O(\log P*)$ calls to the oracle, and
the final binary search looks in a region that is of size
$O(\log P^*)$, this algorithm will make $O(\log P^*)$ uses of
the oracle.

Since it will take of $O(n)$ time to set up this
additional constraint initially (due to the $n$ different
types of vehicles and their corresponding selling prices),
the total runtime of our procedure is $O(n + \log n)$.

\subsection*{(c)}
We can formulate the desired optimization problem into
the following LP (not in standard form):
\begin{displaymath}
\begin{matrix*}[l]
\min              & {\displaystyle \qquad \sum_{j = 1}^{n}b_jy_j }\\
\vspace{-0.5em}\\
\text{subject to} & {\displaystyle \qquad \sum_{j = 1}^{n}a_{ij}y_j \ge \frac{c_i}{2}},
                    & \forall i           &\hspace{-0.5em}\in [1, n]\\
\vspace{-0.5em}\\
                    & \qquad y_j \ge 0,
                    & \forall j           &\hspace{-0.5em} \in [1, n]
\end{matrix*}
\end{displaymath}

where $y_j$ is the price of resource $j$. Upon closer inspection,
we realize that the dual to this LP is the same as the LP in
part (a), with the only change being that the maximization objective
is now multiplied by $1/2$. By strong duality, we know then that
the solution to our minimization LP is equal to the
optimal solution for the dual maximization problem,
which is $\frac{P^*}{2}$. Thus,
the lowest feasible price Melon can pay such that the seller is
satisfied is $\frac{P^*}{2}$.

\end{document}

