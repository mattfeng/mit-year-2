\documentclass{6046}

\usepackage{amsfonts}
\usepackage{amsmath}
\usepackage{bbm}

\author{Matthew Feng}
\problem{8}
\collab{James Lin}

\begin{document}

% =========================
% ======= PROBLEM 1 =======
% =========================

\section*{Problem 1}

\paragraph{Subproblems.}
Let $D(i, j)$ denote the minimum number of uses
of the machine such that using only the first $i$ boxes,
Ben can shrink or grow his boxes in the way described in the
problem such that the volumes of those first $i$ boxes sums to $j$.

These subproblems have optimal substructure because
with $D(i, j) = d^*$ optimal, if there exists
a more optimal solution $d'$ to any subproblem $D(i - 1, j - v_i)$
(where $v_i$ may be any legal volume of box $i$), then
we may simply use that solution instead, achieving a lower
value for $d^*$, violating the notion that $d^*$ was optimal.
Thus no such subsolution $d'$ can exist.

\vspace{-1em}
\paragraph{Relate.}
WLOG, order the dimensions of the boxes such that $x_i \le y_i \le z_i,
\forall i \in [1, n]$.
Our subproblems are related by the following recurrence:

\[
D(i, j) = \min_{x\,\in\,[1, V^{1/3}]}
\left\{
D(i - 1, j - (x)(x + y_i - x_i)(x + z_i - x_i)) + |x_i - x|
\right\}
\]

with base cases
\begin{align*}
D(0, 0) & = 0,\\
D(i, 0) & = \textbf{None},\ \forall i > 0,\\
D(i, j) & = \textbf{None},\ \forall j < 0,
\end{align*}

where the $\min$ ignores $\textbf{None}$ or outputs $\textbf{None}$
if it takes no arguments (i.e. everything it is trying to $\min$
over is also $\textbf{None}$).

In other words, $D(i, j)$ looks at all possible ways to modify
box $i$, and picks the method with the minimum uses of the machine,
provided such a method exists. We are able to bound
the number of modifications we make to box $i$
because we are targeting a specific volume $V$,
and so the smallest dimension does not need to
exceed $V^{1/3}$.

\vspace{-1em}
\paragraph{DAG.}
Since the volume $j$ decreases with every recursive
call, and $i$ decreases as well, every subproblem
only depends on smaller subproblems, and thus the
dependency graph is a DAG. This means that
the problems can be solved bottom-up in an 
efficient manner using memoization.

\vspace{-1em}
\paragraph{Evaluate.}
To solve the original problem, we need to compute the value of $D(n, V)$.

\vspace{-1em}
\paragraph{Analyze.}
The running time of this algorithm can be computed using
the formula
\[
    (\text{number of subproblems}) \times (\text{cost per subproblem}).
\]

There are $O(nV)$ subproblems, and each subproblem looks at
$O(V^{1/3})$ sub-subproblems, and so the total runtime is $O(nV^{4/3})$.

% =========================
% ======= PROBLEM 2 =======
% =========================

\section*{Problem 2}

\paragraph{Subproblems.}
Let us denote the set $C = \{c_i\}$, $\forall i \in [1, n]$
be the set of all colors we may use (i.e. the colors
for each of the different Martian colonies), and let
$A[1:i]$ (inclusive, one-indexed) be the current
coloring of the array, where $A[i] \in C$ denotes
the color at index $i$. Let $DP[1:i]$ represent
the optimal coloring, with $DP[i]$ denoting the
optimal color at index $i$.

Let $D(i, c)$ denote the minimum cost needed to
create peace on Mars for subarray $A[1:i]$,
under the condition that $DP[i] = c$. Each
$D(i, c)$ will be a subproblem in our dynamic
program.

Now, we need to demonstrate that our subproblems
have optimal substructure. Suppose then, that
for subproblem $D(i, c)$ with optimal cost
$d^* = D(i, c)$, there was some coloring
with cost $D'(i - 1, c')$ that was lower than
$D(i - 1, c')$. Then we can simply substitute
$D'(i - 1, c')$ for $D(i - 1, c')$ to obtain
a cost $d' < d^*$, which meant that
$D(i, c) = d^*$ was never optimal. Thus,
the optimal coloring must have optimal subcolorings
as well.

\vspace{-1em}
\paragraph{Relate.}
We have the following recurrence relation:
\[
    D(i, c) = \min_{c' \in C}
        \left(D(i - 1, c') +
        a \mathbbm{1}_{c \ne c'} + 
        b \mathbbm{1}_{c \ne A[i]}
        \right),
\]
where $\mathbbm{1}_{*}$ is the indicator random
variable for $*$, i.e. $1$ if $*$ is true,
and $0$ otherwise.

We also have the following base cases:
\begin{align*}
    D(1, c) & = 0, \forall c \in C,
\end{align*}
i.e. for a single cell, there is no cost for peace.

In other words, we guess that the optimal color of the
last cell is $c$; based on that guess, we want to
find the minimum cost of the remaining $i - 1$ cells,
trying all colors for the $(i - 1)$th cell. If the
$i$th and $(i - 1)$th cells differ in color, we must
build a wall; likewise, if we need to change the color
of the $i$th cell, we need to add that cost as well.
We don't need to consider the cost of changing the color
of the $(i - 1)$th cell, as it will be covered in the
recursive call.

\vspace{-1em}
\paragraph{DAG.}
Our recurrence is a directed acyclic graph (which
allows for efficient computation)
because each subproblem $D(i, c)$ only
depends on subproblems with a smaller index $i -1 $.

\vspace{-1em}
\paragraph{Evaluate.}
Our original problem becomes finding the value of
\[
    \min_{c \in C}D(n, c)
\]
as the final color may be any of the choices in $C$.

\vspace{-1em}
\paragraph{Analyze.}
The number of subproblems in our dynamic program
is $O(nm)$ since a subproblem exists for each
(index, color) pair. Each subproblem requires
$O(m)$ time to solve, as it needs to iterate over
the $m$ different colors in $C$. Overall, then,
our algorithm has a runtime of $(\text{no. of subproblems})
\times (\text{cost per subproblem})$ which equals $O(nm)
\times O(m) = O(nm^2)$, as desired.

% =========================
% ======= PROBLEM 3 =======
% =========================

\section*{Problem 3}
\paragraph{Subproblems.}
Let $F_v$ denote the coefficient of fun for employee $v$,
and let $v.c$ denote the set of children of $v$, i.e. the nodes
$u$ whose boss $\pi(u) = v$. Let $D(v)$ denote the 
maximum sum of the cofficients of fun (we
will use the terminology ``maximum fun'' from now on)
for the subtree
rooted at $v$ (including $v$), subject to the constraint
that if employee $x$ is invited, none of $y \in x.c$
are invited, and likewise that $x$ is not invited if
any of $y \in x.c$ are invited.

Suppose that the our subproblems didn't have optimal substructure.
That means that although $D(v) = f_v^*$, the maximum fun possible,
that one of its subproblems isn't optimal. That means there is
some subtree of $v$ rooted at $u$ such that $D(u) < f_{u}^*$,
where $f_u^*$ is the maximum fun of the subtree rooted at $u$.
But this would be that we could invite the employees whose
coefficients constitute the optimal sum $f_u^*$ and have
a fun greater than $f_v^*$, which is a contradiction. Thus,
the fun for each of the subproblems must also be optimal for $D(v)$
to be optimal.

\vspace{-1em}
\paragraph{Relate.}
Our subproblems are related by the following recurrence:
\[
    D(v) = \max
    \left\{
        F_v + \sum_{u \in v.c}\sum_{w \in u.c} D(w),
        \sum_{u \in v.c} D(u)
    \right\}
\]

with base case $D(v) = F_v$ if $v$ is a leaf.

Essentially, the recurrence states that the maximum fun
for the tree rooted at $v$ has two cases: whether $v$
is invited or not. If $v$ is invited, we account for $v$'s
fun, and then sum over the maximum fun for each of $v$'s
grandchildren (since the children cannot be invited).
On the other hand, if $v$ is not invited, then we can
invite $v$'s children, so we take the sum of the
maximum fun for each of $v$'s children.

\vspace{-1em}
\paragraph{DAG.}
The subproblem dependency graph forms a DAG because the value
of $D(v)$ depends only on its children and grandchildren,
never an ancestor. Thus, these subproblems are able to be
computed efficiently via memoization.

\vspace{-1em}
\paragraph{Evaluate.}
We can find the maximum fun possible by computing $D(b)$, where
$b$ represents Gill Bates. We can compute $D(b)$ with
either a top-down or a bottom-up approach (i.e. start from
leaves vs. start from root).

\vspace{-1em}
\paragraph{Analyze.}
For each of the subproblems $D(v)$, the cost of that subproblem
is
$$|v.c| + \sum_{u \in v.c} |u.c|,$$
i.e. the sum of the number of children and number
of grandchildren of $v$.
That means the total cost of the algorithm is

$$\sum_v(|v.c| + \sum_{u \in v.c} |u.c|).$$

Since each node can only be the child/grandchild
of a single other node, both terms of the outer
sum is bounded $O(n)$. Thus the total runtime of the
algorithm is also $O(n)$.

\end{document}

