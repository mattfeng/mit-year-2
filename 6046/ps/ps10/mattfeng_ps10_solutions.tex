\documentclass{6046}

\usepackage{amsmath}
\usepackage{enumitem}

\author{Matthew Feng}
\problem{10}
\collab{None}

\begin{document}

\section*{Problem 1}
\subsection*{(a)}
We note that the maximum of routers that
may remain on in any $4 \times 4$ grid,
no matter how many routers are available,
is $9$.

Thus, we only need to check all subsets of the
$n$ routers that are size $9$ or less. There
are $O(\binom{n}{9})$ possible subsets, and
for each subset we need to do maximum of $O(n^2)$
work to check if any pairs of points intersect,
which yields in total a polynomial time algorithm
in the number of routers.

For the later parts, we will adjust this
thinking to considering routers that are
completely contained within the $4 \times 4$
area; in that case, a maximum of $4$ routers
be on at once in a $4 \times 4$ area.
However, the algorithm is still polynomial
in $n$ because the size of the subsets
we need to enumerate is bounded.

\subsection*{(b)}
The divide and conquer algorithm doesn't work
because the optimal set of routers may
overlap blocks of $4 \times 4$.

In the first algorithm method, since
we may select routers that overlap into other
grid cells, we may force out the selection
of some optimal routers based on the order
we iterate over the grid cells (or
if we don't bother checking router
coverage, the router coverage will overlap).

In the second algorithm method,
if we only consider routers completely contained
within a grid cell, then potentially optimal routers
that cross grid cells will not be selected.

\subsection*{(c)}
We can take the approach of the second algorithm, 
that is only selecting routers that are completely
contained within a grid cell, and combine it with
a randomly offset grid of width $4$ along both $x$ and $y$
axes, such that the expected number of
routers that are selected to
remain on is $\dfrac{|C^*|}{4}$,
where $C^*$ is the optimal set of
routers.

To see this, consider the optimal set $C^*$
of routers. With a randomly offset grid,
the probability that a router $c \in C^*$
has its unit disk of coverage completely
contained within a grid cell of width $4$
is $\frac{1}{4}$, since both the $x$ and $y$
axes must not intersect the disk. In particular,
suppose the disk is centered at $(1, 1)$.
For a grid of width $4$, the only offsets
that don't result in a grid intersecting
with the disk are within the box bounded
by $(2, 2)$ and $(4, 4)$ (since the grid
repeats, we don't need to consider any
other offsets). Thus, a randomly offset
the expected number of disks in the
optimal set lie entirely within a grid cell
is $\dfrac{|C^*|}{4}$ by linearity of expectation
(and indicator variables), as desired.

\subsection*{(d)}
We can generalize this algorithm into a PTAS by
increasing the size of our grid cells. Suppose
we have a grid where the width of each cell is
$k$. Then the probability that a unit disk
is not intersected by a randomly offset grid
is
\[
    \frac{(\frac{k}{2} - 1)^2}{(\frac{k}{2})^2} = (1 - \frac{2}{k})^2.
\]

Thus, the expected fraction of the optimal
solution we will achieve is $(1 - \frac{2}{k})^2$.
Setting $1 - \epsilon = (1 - \frac{2}{k})^2$.
Solving for $k$ yields
\[
    k = \frac{2}{1 - \sqrt{1 - \epsilon}}.
\]

With at most $(\frac{k}{2})^2$ routers per cell,
the running time of any algorithm in the PTAS
is $O(n^k)$ (since there are constant number of
grid cells, and the maximum size of a subset
for a cell is $(\frac{k}{2})^2)$).



\section*{Problem 2}
\subsection*{(a)}
With a single query, the case where
$i$ and $j$ have not spoken with
each other may be attributed to one of two reasons:
\vspace{-1em}
\begin{enumerate}[noitemsep]
    \item $i$ and $j$ are in the same agency.
    \item $i$ and $j$ are in different agencies, but by chance
    are not in the $\frac{3n}{4}$ spies that spoke to each other.
\end{enumerate}
\vspace{-1em}
Since we cannot distinguish between to the two reasons
with a single query, and the two reasons yield
opposite answers, we cannot be correct on all inputs.

\subsection*{(b)}
We can conclude that spies $i$ and $j$ are from the
same agency, since they cannot be in the agencies
of $m$ and $n$ whom are part of different agencies,
and there are only three agencies in total.

\subsection*{(c)}
The probability that $m$ and $n$, sampled
uniformly at random, are in different
agencies, are not in the same agency as $i$ and $j$,
have talked to each other, and have
each spoken with $i$ and $j$, given that
$i$ and $j$ are in the same agency, is
at least
\[
    (\frac{2}{3})^3 \times (\frac{3}{4})^5 = \frac{9}{128} > \frac{1}{36},
\]
because there is
\vspace{-1em}
\begin{enumerate}[noitemsep]
    \item $2/3$ probability $m$ and $n$ are in different agencies,
    \item $2/3$ probability $m$ and $i$ are in different agencies
          (which implies $m$ and $j$ as well),
    \item $2/3$ probability $n$ and $i$ are in different agencies,
    \item at least $3/4$ probability for each pair in
          $\{(m, n), (m, i), (m, j), (n, i), (n, j)\}$ to have spoken with each other.
\end{enumerate}

\subsection*{(d)}
Given two spies $i$ and $j$, there are two cases
which we need to consider:
\vspace{-1em}
\paragraph{Case 1. $i$ and $j$ have talked to each other.}
In this case, $i$ and $j$ must be in different agencies,
so we output {\tt DIFFERENT}.

\vspace{-1em}
\paragraph{Case 2. $i$ and $j$ have {\it not} talked to each other.}
In this case, we use the construction described
in part (b) which samples two other
spies $m$ and $n$ uniformly at random
to see if $i$ and $j$ are part of the
same agency. With probability of at least
$\frac{1}{36}$, we will find the construction,
letting us conclude that $i$ and $j$ are in the
same agency, and we output {\tt SAME}. Otherwise, we
output {\tt DIFFERENT}.

Our algorithm has the desirable property
that if $i$ and $j$ are in different agencies,
then we will always output the correct answer
{\tt DIFFERENT}, since the construction leading
the output of {\tt SAME} is impossible to sample.
However, if $i$ and $j$ are in the same agency,
there is at least a $\frac{1}{36}$ probability
that we will output the correct answer.

With this property (that the output for
one possible outcome is always correct, and the
other is probabilistically correct), we
can simply run this algorithm many times until
we get the desired accuracy.

In particular, instead of sampling
$(m, n)$ only once, we sample $k$ times.
Each time, there is at most $\frac{35}{36}$
probability that the construction does not
exist, but $i$ and $j$ are in the same agency.
There is only a probability of at most
$(\frac{35}{36})^k$ for this to occur all
$k$ times (we know that if the construction
exists for any of the $k$ samples, $i$ and $j$
are in the same agency). Thus, the probability
of outputing the incorrect answer is
$(\frac{35}{36})^k$. For $k \ge 25$,
the probability of incorrectness is below
$1/2$ (i.e. probability of correct output
is greater than $1/2$, as desired). $k$ is
a constant, and checking to see if the
construction in (b) exists takes a constant
number of queries, which means
we can determine whether or not $i$ and $j$
belong to the same agency with a constant
number of queries.


\end{document}

