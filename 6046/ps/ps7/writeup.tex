\documentclass{6046}

\usepackage{float}
\usepackage{algorithm}
\usepackage{algpseudocode}

\author{Matthew Feng}
\problem{7}
% \problem{A} means Problem Set A.
\collab{James Lin}
% or give names, e.g., \collab{Alyssa P. Hacker and A. Student}

\begin{document}

\section*{Problem 1}
\subsection*{(a)}
For our deterministic algorithm to work as desired,
the best we can do is return
``SORTED'' immediately if more than $\frac{9}{10}n$ elements
are indeed sorted (since we don't care what we output
if the array is mostly sorted but not fully sorted),
and return ``UNSORTED'' immediately if
we find more than $\frac{1}{10}n$ elements to be unsorted.
This is the best because if we don't check $\frac{9}{10}n$
elements, the array could possibly be unsorted, yet
we would return ``SORTED''. Thus, we need to check
$\frac{9}{10}n$ elements for a sorted array $A$, which
is $\Omega(n)$ runtime.

\subsection*{(b)}
Consider the array $A = [1, 2, 3, ..., n]$ that is rotated by
$\frac{n}{2}$ indices, so that the resulting array $A$ is
$[n/2, n/2 + 1, ..., n - 2, n - 1, n, 1, 2, 3, ... n/2 - 1]$.
$A$ is not mostly sorted, since only by removing (at least) 
$n/2$ elements can we obtain a sorted array. For $n - 4$
indices, $A[i] < A[i + 1] < A[i + 2]$. Only for
two triples, namely $(n - 1, n, 1)$ and $(n, 1, 2)$,
will checking three consecutive indices reveal that
$A$ is unsorted.

For this adversarial example, if we randomly select an index
$i$ from $0$ to $n - 3$, the probability that $A[i], A[i + 1]$,
and $A[i + 2$ will be sorted is $\frac{n - 4}{n - 2}$. Our algorithm
fails if it outputs ``SORTED''. The probability that this occurs
is $p = \Pr[fail] = (\frac{n - 4}{n - 2})^k$, since we have $k$ independent
iterations of testing indices, and all must pass.
Our goal is to have $p < \frac{1}{3}$ for all values of $n$. This
is not possible unless $k = \Omega(n)$. To see this, suppose
the contrary: that $k = o(n)$ can satisfy that $p < \frac{1}{3}$.
This means that $\forall \epsilon > 0, \exists N$
such that $\forall n > N$, $k < \epsilon n$, and
$(\frac{n - 4}{n - 2})^k = (1 - \frac{2}{n - 2})^k < \frac{1}{3}$.
However, this is false. $(1 - \frac{2}{n - 2})^k$ is minimized when
$k$ is maximized; that is, $p$ is lower-bounded by
$(1 - \frac{2}{n - 2})^{\epsilon n} \approx (\frac{1}{e^2})^\epsilon$.
For very small $\epsilon$, this value is larger than $\frac{1}{3}$.
Thus, $k$ cannot be $o(n)$, which means it must be $\Omega(n)$.

\subsection*{(c)}
Let us consider the point at which the execution of
$i_1 \leftarrow$ {\sc Binary-Search($A$, $x_1$, $0$, $n$)}
diverges from  $i_2 \leftarrow$ {\sc Binary-Search($A$, $x_2$, $0$, $n$)}. The only point
at which this is possible is if $x_1 < A[median]$ and
$x_2 \ge A[median]$, or vice versa. Since $i_1 < i_2$, however,
it must be the case that $x_1 < A[median]$ and $x_2 \ge A[median]$,
or $x_1 < A[median] \le x_2 \implies x_1 < x_2$.

\subsection*{(d)}
Suppose (for the sake of contradiction) that $A$ is {\bf not}
mostly sorted, yet more than $\frac{9}{10}n$
indices can pass {\sc Binary-Search-Test}. We can order these indices
in increasing order $\{i_1, i_2, i_3, ..., i_{9n/10}, i_{9n/10 + 1}, ...\}$.
Then, by the property demonstrated in part (c), we know that
$A[i_1] < A[i_2] < A[i_3] < ... < A[i_{9n/10}] < A[i_{9n/10 + 1}]$.
But that means that the array $A$ is mostly sorted.
This is a contradiction, and thus no more than $\frac{9}{10}n$ can
pass {\sc Binary-Search-Test} for an array $A$ that is
not mostly sorted.

\subsection*{(e)}
We can now use {\sc Binary-Search-Test} as a subroutine for our
algorithm. The algorithm is as follows:

\begin{algorithm}[H]
  \caption{Check if an array is sorted using randomization}\label{RandCheck}
  \begin{algorithmic}[0]
    \Function{Check-Sorted}{$A$}
    \For{100 iterations}
        \State generate random index $i \in [0, n - 1]$
        \State success $\leftarrow$ \Call{Binary-Search-Test}{$A$, $i$}
        \If{success $\not=$ True}
            \State \Return ``UNSORTED''
        \EndIf
    \EndFor
    \State \Return ``SORTED''
    \EndFunction
  \end{algorithmic}
\end{algorithm}

The main idea behind the algorithm is that if {\sc Binary-Search-Test}
fails for any index $i \in [0, n - 1]$, then $A$ is not sorted.
Similarly, the more times {\sc Binary-Search-Test} succeeds, the
more probable that $A$ is sorted.

The algorithm calls {\sc Binary-Search-Test} with a random index;
if {\sc Binary-Search-Test} fails, then we know that the array
$A$ cannot be sorted, and so we return ``UNSORTED.'' However,
if our tests succeed a sufficient number of times, then we
can be fairly confident that the array is sorted, and thus
we return ``SORTED'' with some probability that we will
be incorrect.

Formally, we can argue the correctness of this algorithm
through casework. There are three cases:
\vspace{-1.5em}

\paragraph{Case 1.} $A$ is sorted.
\vspace{-0.75em}

Since $A$ is sorted, {\sc Binary-Search-Test} will always succeed,
and thus we will always return ``SORTED'', as desired.
\vspace{-1em}

\paragraph{Case 2.} $A$ is mostly sorted, but not (completely) sorted.
\vspace{-0.75em}

In this case, our algorithm may return ``SORTED'' or ``UNSORTED'',
but we don't care what the output is; thus, this case is
satisfactory as well.
\vspace{-1em}

\paragraph{Case 3.} $A$ is not mostly sorted.
\vspace{-0.75em}

First, if $A$ is not mostly sorted, our algorithm fails
if it outputs ``SORTED.'' This means that our algorithm
fails if all $100$ calls to {\sc Binary-Search-Test} succeed.

Next, since $A$ is not mostly sorted, by the fact shown in (d) that
{\sc Binary-Search-Test} will succeed on no more than
$\frac{9}{10}n$ indices, we can say that
for a not-mostly-sorted array, each individual call to
{\sc Binary-Search-Test} will succeed {\bf with probability $\le \frac{9}{10}$}.

Thus, the chance that {\it every call} of
{\sc Binary-Search-Test} on $A$ succeeds, for $k$ calls,
is bounded above by $(\frac{9}{10})^k$. For a large
enough constant $k$ (in particular, $k = 22$), this probability
drops below $\frac{1}{10}$. In other words, the probability that
our algorithm fails for this case is less than $\frac{1}{10}$,
for all $k \ge 22$. Since we set $k = 100$, the probability that
our algorithm returns the incorrect output is well below $\frac{1}{10}$.

\vspace{1.5em}

Since $A$ must fall in one of these three cases,
and that our algorithm succeeds all the time
for cases $1$ and $2$, and fails
with very little probability for case $3$,
our algorithm fits the specifications. Furthermore,
since we make a constant number of calls to {\sc Binary-Search-Test},
which runs in $O(\log n)$ time, our algorithm runs in
$O(1) \times O(\log n)$ time complexity, which remains $O(\log n)$,
as desired.

\section*{Problem 2}
\subsection*{(a)}
For both of Melon's algorithms, if the length of the random walk
is long enough so that every node is visited, then the edges
that are selected edges must form a spanning tree.

For both algorithms, we notice that the number of edges
we admit is $|V| - 1$.

Then, for algorithm 1, every node $u$ other
than the start node $s$ is the destination in exactly one
edge $(\,\cdot\,, u)$ in the edges we select ($s$ must
be the source for at least one of these edges, i.e.
$(s, \,\cdot\,)$). Since we visit every vertex, there must be exactly
$|V|$ distinct nodes adjacent to the edges we have selected.
Since there are $|V|$ nodes and $|V| - 1$ edges, the edges
we have selected must form a tree. Additionally, since
all $|V|$ nodes in $G$ are accounted for, the edge-set
we have selected via algorithm 1 must form a spanning tree.

A similar analysis works for algorithm 2. In algorithm 2,
every node $u$ other than the final node $s$ must be
the source in exactly one edge $(u, \,\cdot\,)$, and $s$
must be the destination of one these edges $(\,\cdot\,, s)$.
Again, all $|V|$ distinct nodes are adjacent to our selected
edge-set, which contains $|V| - 1$ edges. Thus our edge must
be a tree, and it spans all $|V|$ vertices in $G$.

\subsection*{(b)}
We notice that for any particular random walk
$w = (u_0, u_1, ..., u_k)$ that covers all the nodes in $G$, 
the spanning tree constructed using algorithm 1 is the same
as the spanning tree constructed using algorithm 2, if
the random walk was performed in reverse as $w_{rev} =
(u_k, u_k-1, ..., u_1, u_0)$. To see this, for algorithm 1,
we exclude $u_0$ from the set of vertices we perform our
union over; we exclude $u_0$ in algorithm 2 as well. Next,
if $u_i \rightarrow u_{i + 1}$ is the first transition in $w$
where $u_{i + 1}$ is visited, then $u_{i + 1} \rightarrow u_{i}$
must be the last transition from $u_{i + 1}$ in $w_{rev}$. Thus,
the set of edges we select from algorithm 1 and 2,
using $w$ and $w_{rev}$ respectively, will be the same. What
remains to show is that the probability that $w_{rev}$ is 
generated from a random walk is the same as the
probability that $w$ is generated from a random walk. The
probability that we generate $w$ is $\Pr[u_0] \times
\prod_{i = 1}^k \Pr[u_i | u_{i - 1}] = 1/|V|
\times (1/d)^k$. The probability
that we generate $w_{rev}$ is $\Pr[u_k] \times
\prod_{i = 0}^{k - 1}\Pr[u_i | u_{i + 1}] = 1/|V|
\times (1/d)^k$. We see that the probabilities are
equivalent, and thus the distribution of spanning
trees generated by algorithms 1 and 2 must be the same
after the same number of steps.

\subsection*{(c)}
To show that the uniform distribution $u = [1/n, 1/n, ..., 1/n]$
($u$ is a row vector) is stationary for a
Markov chain with $n \times n$ size
transition matrix $W$, we need to show that $uW = u$. If
we adopt the ``row $\times$ matrix'' perspective of matrix
multiplication, we see that $uW$ is equivalent
to $\sum_{i = 1}^{n} u_i \times (\text{row $i$ of $W$}) = 
\sum_{i = 1}^n (1/n) \times (\text{row $i$ of $W$}) = 
(1/n) \sum_{i = 1}^n (\text{row $i$ of $W$}) =
(1/n) [1, 1, ..., 1] = u$.

\subsection*{(d)}
First, view every edge as two different halves: an outgoing, source
``half,'' and an incoming, destination ``half.'' Every
edge must have both parts. For a digraph
$G = (V, E)$, if the out-degree of every node is
exactly $d$, then there are exactly $|V|d$ outgoing ``halves''.
Consequently, there must be exactly $|V|d$ incoming ``halves''.
Since the in-degree of any node must be $\le d$, the
total number of incoming ``halves'' that can exist in $G$
is $\le |V|d$, with equality if and only if every node has
an in-degree of $d$. Thus, since we have exactly $|V|d$ incoming
halves, this implies that every node has an in-degree of $d$.

\subsection*{(e)}
If we can show that any node in the Markov chain as described above
(i.e. on the node set $V(G) \times \mathcal{T}$) can have
in-degree at most $d$, and we know that every node in the Markov
chain has out-degree of $d$ (since every node $v \in V(G)$
is connected to $d$ neighbors), then by Part (d) we know
that the in-degree of every node must be $d$.

Suppose, for the sake of contradiction, 
that $(u, T)$ has an in-degree $\hat d$ greater than $d$. We note that
we can ``reverse'' any transition $(v_i, T_i) \rightarrow (u, T),
\forall i \in [1, \hat d]$ (the $v_i$'s may not be distinct, but
the $T_i$'s will be); that is, we can delete
$(u, v_i)$ in $E(T)$ to form $T_u$ and $T_{v_i}$,
two trees rooted at $u$ and $v_i$, respectively, and reattach
$u$ to its original parent $\pi_i(u)$ (which may be $v_i$) in $T_i$. This means
that $u$ has $\hat d$ distinct valid parents, one for each $T_i$
rooted at $v_i$. However, in order for $\pi_i(u)$ to be a parent,
$\pi_i(u)$ must have been connected to $u$ in $G$, i.e.
$(\pi_i(u), u) \in E(G)$. But this means that $u$ has $\hat d > d$
neighbors in $G$, which is impossible.

Thus, the number of transitions that
end at $(u, T)$ is cannot be greater than $d$. This is
exactly what we wanted to show; as such, this fact implies,
by Part (d), that the in-degree of every node in the
Markov chain is exactly $d$.

Now, we need to show that the transition/walk matrix $W$
formed by this Markov chain is doubly stochastic, that is,
every row and every column sums to $1$.
$W_{ij}$ may represent the transition from node $i$ to node $j$
(we are using left matrix multiplication). Then every row $i$
in $W$ represents the transition distribution across
all other nodes, starting at $i$. Since there are
$d$ outgoing edges from $i$, each with probability $1/d$,
for every $i$, the sum of every row is $1$.
Likewise, the column $j$ in $W$ represents all the 
edges incoming to $j$ and their respective probabilities.
We showed that the in-degree for any node $j$ must be $d$,
and each transition always has probability $1/d$, and
thus all the columns must also sum to $1$.

Thus, since $W$ is doubly-stochastic and converges to
a unique, stationary distribution, that distribution
must be uniform across all nodes $V(G) \times \mathcal{T}$,
which yields a uniform random distribution over $\mathcal{T}$,
as desired.

\end{document}

