\documentclass{6046}

\author{Matthew Feng}
\problem{3}
\collab{Priscilla Wong, Katherine Xiao}

\begin{document}


\section*{Problem 1}
\subsection*{(a)}

For any deck ordering $D$ that Ben selects, the adversarial
audience, having knowledge of $D$, can force Ben into using
$\Omega(n^2)$ draws.

First we notice that having $D = (\textit{all Rs})(\textit{all Bs})$
(or vice versa) and an audience ordering of $BRBR...BR$ (i.e. alternating
card color requests) forces Ben to use $\Omega(n^2)$ deck draws,
because for every request, he has to move $n - \ceil{(i - 1) / 2}$ cards from
the top of the deck to the bottom of the deck for the $i$th
request. With $n$ requests, the sum of draws is $\Omega(n^2)$.

Now, the audience can force the ordering of the deck to be the
above case while still ensuring that $\frac{n}{2}$ cards of 
both colors remain in the deck. To do this, the audience
splits the deck into two halves --- the front half $X$ and
the back half $Y$. We count the number of red and black cards
in $X$, denoted $r_X$ and $b_X$ respectively, as well as
the number of red and black cards in $Y$, denoted $r_Y$ and $b_Y$,
respectively.

WLOG assume we have $b_X > r_X$, which implies that $r_Y > b_Y$.
We notice that if we order $r_X + 1$ red card requests
at the front of our audience, then we will force $b_X$ black cards to the back of
Ben's deck. $Y$ will now be at the top of the deck. Likewise,
we can have $b_Y$ black card requests follow the $r_X + 1$
red card requests, which will force $r_Y$ red cards to the
back of Ben's deck. Because $b_X > n / 2$ and $r_Y > n / 2$,
and they are blocked together, we have constructed the worst case
deck ordering for Ben, which we know is $\Omega(m^2)$, where $m$
the number of each color remaining. Since $m = \Omega(n / 2)$, we
have $\Omega(\Omega(n / 2) ^ 2) = \Omega(n^2)$.

Because in the worst case the online algorithm will
require $\Omega(n^2)$ operations, while the offline
optimal algorithm will be able to run in $O(n)$
operations, the competitive ratio of Ben's
deck ordering is $\Omega(n^2) / O(n)$, or never
better than $n$-competitive.

\subsection*{(b)}
Ben can achieve drawing at most $O(n)$ cards for any audience
ordering if he is able to move cards from the bottom of the
deck to the top of the deck and vice versa if he maintains
the heuristic that, WLOG, he always has a red card at the top
of the deck and a blue card at the bottom of the deck. If he
maintains this property, moving a card from the top of the deck
to the bottom of the deck if it isn't red, and moving a card
from the bottom of the deck to the top if it isn't black, then
Ben will always end up with a block of red cards at the top
of the deck and a block of black cards at the bottom of the deck.

WLOG, let us define an inversion as the number of black cards $B$
that are located in the top half of the deck. Every time we move
a card from the front to the back, or the back to the front, we
are fixing one inversion. There are at most $n$ inversions at
the start of the shuffle, and so Ben requires at most $O(n)$
draws from top to bottom or bottom to top in order to fix all
the inversions.

Because Ben has an $O(n)$ strategy for any ordering of the audience,
and the optimal offline strategy is $O(n)$, we know that 
Ben's strategy is $\alpha$-competitive, for some constant $\alpha > 1$.

\section*{Problem 2}
Our data structure $D$ will have multiple fields. First,
we will have a two dimensional matrix $D.coins$, which will
hold the same values as $G$ (i.e. $0$ if no coin, $1$ if coin),
with the addition that if we have visited cell $D.coins[i, j]$,
then $D.coins[i, j]$ is marked with $*$.
$D$ will also have a counter $D.c$ that keeps track
of the number of coins we have yet to collect.

\paragraph{initialize($G$)} We can initialize $D$ in
$O(mn)$ time by copying $G$ into $D.coins$, and while doing so,
counting the number of coins on the grid $G$ and saving that
value to $D.c$. We look at every cell in our grid exactly once,
so initialization takes $O(mn)$ time.

\paragraph{numCoins} We can simply return the value of
our counter $D.c$. This is a $O(1)$ operation.

\paragraph{removeCoin($i$, $j$)} We can remove the
coin at $(i, j)$ by setting $D.coins[i, j] = *$. We also
must decrement $D.c$ by 1. This takes at most $O(1)$ time.

\paragraph{nextCoin($i$, $j$)} We notice that we never
need to explore a cell more than once, because robot paths
never intersect (which was a proof in lecture if I recall correctly).
Thus, our $nextCoin(i, j)$ routine is very simple:
starting from $(i, j)$, scan down the column $j$ to look for coins.
Mark all scanned cells with a $*$. If we don't find any coins, increment
$j$ to scan down the next column starting at row $i$,
and repeat until we find a coin. If we reach $(m, n)$ without
finding a coin, return ``error''. Furthermore, if we encounter
a $*$ when scanning down a column, immediately increment $j$
and start scanning the next column. Because we only explore each
cell at most once, because we don't explore once they are
marked as $*$, in aggregate this routine takes $O(mn)$.

However, we need to demonstrate that this algorithm is correct.
We can see this because it scans all possible locations that
the next coin may be located (to the right and to the bottom),
in a order that finds the leftmost, topmost valid coin (scan from
top to bottom, starting from the left), which is what we desired.

The runtime of our original algorithm was $O(rmn)$ because
each call to $nextCoin$ took $O(mn)$ time; however, because
with data structure $D$, these calls in aggregate cost $O(mn)$,
the total run time of the \textsc{PeelOff($G$)} algorithm is $O(mn)$.

\section*{Problem 3}
\subsection*{(a)}
We can implement the \textsc{Update($i_1$, $i_2$, $x$)} and
\textsc{PieTime()} operations using a \textsc{Union-Find}
data structure represented as a {\bf forest of trees}.

In particular, we note that if we know the ratio $r_{(a,b)}$ between
ingredients $a$ and $b$, and we know the ratio $r_{(b,c)}$ between
ingredients $b$ and $c$, then we know the ratio $r_{(a,c)}$ 
between $a$ and $c$ as $r_{(a,c)} = r_{(a,b)}r_{(b,c)}$. What
this tells us is that we can think of the ingredients as
nodes in an undirected graph $G = (V, E)$, and every time we learn the ratio
between ingredient $i_1, i_2 \in V$, we add edge $(i_1, i_2)$
to $E$.

More importantly, however, we want to keep track of the
{\bf connected components} in our graph $G$. This is because
as long as there is a path between two nodes $i_1$, $i_2$, we
can compute the ratio between the two ingredients through
successive multiplications.

We learned in class how to create a \textsc{Union-Find} data
structure that allows \textsc{Union($x$, $y$)} and
\textsc{Find-Set($x$)} in amortized $O(\alpha(n))$.
We augment this data structure with a counter $C$ that counts
the number of connected components in $G$, and hashset $I$
that keeps track of the different ingredients
we have in our recipe.

To implement \textsc{Update($i_1$, $i_2$, $x$)}, we first check
to see if $i_1, i_2 \in I$, which we can do in expected $O(1)$
time. If an ingredient is not yet in our graph, we call
\textsc{Make-Set($i$)} for that ingredient, add $i$ to $I$, and
increment $C$ for new ingredient we add,
all of which are $O(1)$ operations. Next, we check that $i_1$
and $i_2$ are not part of the same connected component with
by checking that \textsc{Find-Set($i_1$)} $\neq$
\textsc{Find-Set($i_2$)}. If $i_1$ and $i_2$ are in
separate connected components, then we decrement $C$ by 1,
because there will be one fewer connected component after the
union. Calling the \textsc{Find-Set($i$)} operation twice
costs $O(\alpha(n))$ amortized. Finally, we call
\textsc{Union($i_1$, $i_2$)}, which will connect the
connected components $i_1$ and $i_2$ belong to, together. This
will also cost $O(\alpha(n))$ amortized.

To check if we are able to bake the pie via \textsc{PieTime()},
we only need to check if the number of connected components
we have is $1$. In other words, return whether or not
$C = 1$. This takes $O(1)$ time.

Note that this implementation does not store the actual ratio
information; the data structure only tells us that we {\it have 
enough knowledge} to compute the ratio between two ingredients 
--- it doesn't help us compute that ratio (which is left for 
part (b)).

\subsection*{(b)}
In order to implement \textsc{GetRatio($i_1$, $i_2$)} in
amortized $O(\alpha(n))$ time, we augment every node 
into our forest of trees with a value $node.r$, which
stores the ratio between the ingredients $i$ and its
parent $i.parent$, i.e. $i.r = q(i) / q(i.parent)
= \rho(i, i.parent)$. Our goal is to use
the traversal up the tree during the calls to
\textsc{Find-Set($i_1$)} and \textsc{Find-Set($i_2$)} to
compute $r_1 = \rho(i_1, ref[i_1])$ and $r_2 = \rho(i_2, ref[i_2])$, 
and, assuming $ref[i_1] = ref[i_2]$ (if not, just return
that we don't know the ratio between the two), computing 
and returning back to the user $\rho(i_1, i_2) = r_1 / r_2$.

In order for our data structure to function correctly,
we need to ensure that for every ingredient, we are
able to compute $i.r = \rho(i, i.parent)$. We will show
that we can update this value throughout any call
to \textsc{Find-Set} and \textsc{Union}.

When \textsc{Union($i_1$, $i_2$, $x$)} is called, WLOG assume
$i_2 = i_1.parent$, due to the heights of their corresponding
subtrees. Then $i_1.r = q(i_1) / q(i_2) = x$, and $i_2.r$ is
unchanged.

When we use \textsc{Find-Set($i$)} internally, we need to
make sure that we update the ratios of the nodes
along the path we traverse, because we reassign their parents
to flatten the tree. This is easily doable, however. Suppose
our path up to $ref[i]$ is $(i_1, i_2, ..., i_k)$, such that
$i_n.parent = i_{n+1}, \forall n \in [1, k - 1]$. As we
traverse up the path, we keep track of $r_n = q(i_1) / q(i_n)$ for
every node in the path, and compute $r_k = q(i_1) / q(i_k)$ at
the end of the traversal. We are then able to calculate
and update the new values for $i_n.r = q(i_n) / q(i_k) =
r_k / r_n$ at the same time we update the parent pointers. Since
we are simply adding $O(1)$ operations to each step of the
\textsc{Find-Set} and \textsc{Union} function calls, we don't change the 
asymptotic time complexity of those functions, and so we
are able to implement \textsc{GetRatio($i_1$, $i_2$)} in
amortized $O(\alpha(n))$.

\end{document}

